
\chapter{Introduction}
{
	\section{Introduction}
	{
		The integrity of systems files is necessary for an operating system to
		function securely.  Integrity is not generally discussed in terms of
		complete computer systems. Instead, integrity issues tend to be either
		tightly coupled to a particular domain (e.g. database constraints), or
		so broad as to be useless except after the fact (e.g. backups).
		Often, file integrity is determined by who modifies the file or by a
		checksum. Our appoarch is to focus on a general model for representing
		the internal integrity of a file. Even if a file is modified by a
		subject with trust or whose modification has a valid checksum, it may
		not meet the specification of a valid file. An example would be a
		\texttt{/etc/passwd} file with no user assigned a user id of 0. This
		repository stores the source code of a program called Guardian that
		provides a means to specify and enforce what the contents of a valid
		file should be.
		Guardian can be used to specify the format and valid properties of
		system configuration files, SSH-key files and others.
		
		Guardian generates the C source code for a program that will determine
		if the content of a given file matches the program's specification.
		The file specification itself is
		described through grammar rules and assertions made in Guardian's
		programming language. The file that contains these grammar rules and
		assertions is expected to be given to Guardian on its command-line
		invocation.
		For ease of organization the entire file specification may span multiple
		files.
	}
}



















