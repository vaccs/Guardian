
\section{All Special Forms}
{
	The list below briefly describes all of the special forms in Guardian's
	programming language, in alphabetical order.
	
	\begin{itemize}
	{
		\item[] \texttt{all!}
			
			This special form expects a boolean list, returns true if all of the values
			are true.
			
		\item[] \texttt{any!}
		
			This special form expects a boolean list, returns true if any of the values
			are true.
		
		\item[] \texttt{crossmap!}
		
			This special form expects a lambda and one or more lists, returns
			the list of results of calling the lambda on each combination
			of elements from each list.
		
		\item[] \texttt{filter!}
		
			This special form expects a lambda and a list, returns the list
			of elements of the given list that caused the lambda to return true.
		
		\item[] \texttt{float!}
		
			This special form can take either an integer or a string, and
			returns that value converted to a float.
		
		\item[] \texttt{int!}
		
			This special form can take either a float or a string, and
			returns that value converted to an integer.
		
		\item[] \texttt{isabspath!}
		
			This special form expects a string path, and returns
			a boolean value indicating if the given path is an absolute path.
		
		\item[] \texttt{isaccessibleto!}
		
			This special form expects two strings, a path and a user, and
			returns a boolean value indicating if the given path is
			accessible (readable) to the given user.
		
		\item[] \texttt{isdir!}
		
			This special form expects a string path, and returns
			a boolean value indicating if the given path refers to a
			directory.
		
		\item[] \texttt{isexecutableby!}
		
			This special form expects two strings, a path and a user, and
			returns a boolean value indicating if the given path is
			executable to the given user.
		
		\item[] \texttt{len!}
			
			This special form will accept either a string, a list, a dictionary,
			a set or a tuple, and returns the integer number of elements in
			the object.
		
		\item[] \texttt{map!}
		
			This special form expects a lambda and one or more lists,
			and returns the list of results from calling the lambda on all
			of the first parameters, all of the second parameters, all of the
			third parameters, etc.
		
		\item[] \texttt{range!}
		
			This special form expects either 1 or 2 integer values,
			and returns the integer list of values ranging from the first to
			the second value. The second value is not included in the range.
			If only one value is provided the range starts at 0 and ranges
			to the given value.
		
		\item[] \texttt{reduce!}
		
			This special form expects a lambda value (\texttt{x}),
			a list value (\texttt{y}) and an initial value (\texttt{z})
			and behaves as described by this python code:
\begin{lstlisting}[language=Python]
	def reduce(x, y, z):
		retval = z
		for e in y:
			retval = x(retval, e);
		return retval;
\end{lstlisting}
		
		\item[] \texttt{sum!}
		
			Expects either an integer list or a float list and returns
			the sum of the values.
	}
	\end{itemize}
	
}











