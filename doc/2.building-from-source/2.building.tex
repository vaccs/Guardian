
\section{Building}
{
	Once all of the dependencies are installed, Guardian can be built. The
	makefile that builds Guardian has several build options described below.
	For most users the default build options are sufficient.
	
	\begin{itemize}
	{
		\item[] {\texttt{\$ export buildtype=release} (default)}\\
		{
			The intended use case for this build type is to build Guardian to be
			ready to be publicly released.
			This build type causes Guardian to be statically-linked and to be
			compiled with all warnings and errors and optimizations enabled.
		}
		
		\item[] {\texttt{\$ export buildtype=test}}\\
		{
			The intended use case for this build type is to test Guardian's
			functionality without
			performing the steps to make it ready to be publicly released.
			This build type causes Guardian to be dynamically linked.
			Certain warnings and errors and optimizations are disabled for
			ease of development.
		}
		
		\item[] {\texttt{\$ export buildtype=debug}}\\
		{
			The intended use case for this build type is for tracking down
			bugs within Guardian or for developing new features for Guardian.
			This build type causes Guardian to be dynamically linked. Certain
			warnings and errors
			are disabled for ease of development, and extra calls to
			\texttt{printf}
			at many points in the code are inserted to aid in debugging.
		}
		
		\item[] {\texttt{\$ export verbose=yes} (default)}\\
		{
			Several large functions must be included in order to support the
			\texttt{-v} (verbosity) command-line flag. This build
			type causes those
			functions to be included in Guardian's executable.
		}
		
		\item[] {\texttt{\$ export verbose=no}}\\
		{
			This build type causes Guardian to build without support for the
			\texttt{-v} (verbosity) command-line flag. This build type
			silently ignores the \texttt{-v} command-line option.
		}
		
		\item[] {\texttt{\$ export dotout=no} (default)}\\
		{
			This build type disables generation of GraphViz diagrams for the
			state machines generated by Guardian.
		}
		
		\item[] {\texttt{\$ export dotout=yes}}\\
		{
			The intended use case for this build type is help track down bugs
			in the state machines generated by Guardian. This build type
			inserts code to generate GraphViz diagrams of the
			regular expression,
			grammar, parser, and tokenizer state machines.
		}
	}
	\end{itemize}
	
	Once the build options have been set, Guardian can be built with the
	\texttt{make} command.
	
	The \texttt{make install} command will by default install Guardian into
	\texttt{\$\{HOME\}/bin}; the installation path can be changed with
	\texttt{make install PREFIX=\$\{INSTALLPATH\}}.
}
















