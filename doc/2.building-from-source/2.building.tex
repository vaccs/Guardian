
\section{Building}
{
	Once all of the dependencies are installed. Guardian can be built. The
	makefile that builds Guardian has several build options discribed below.
	For most users though the default build options are sufficient.
	
	\begin{itemize}
	{
		\item {\texttt{\$ export buildtype=release} (default)}\\
		{
			\lipsum[1]
		}
		
		\item {\texttt{\$ export buildtype=test}}\\
		{
			\lipsum[1]
		}
		
		\item {\texttt{\$ export buildtype=debug}}\\
		{
			\lipsum[1]
		}
		
		\item {\texttt{\$ export verbose=yes} (default)}\\
		{
			\lipsum[1]
		}
		
		\item {\texttt{\$ export verbose=no}}\\
		{
			\lipsum[1]
		}
		
		\item {\texttt{\$ export dotout=no} (default)}\\
		{
			\lipsum[1]
		}
		
		\item {\texttt{\$ export dotout=yes}}\\
		{
			\lipsum[1]
		}
	}
	\end{itemize}
	
	Once the build options have been set, Guardian can be built with the
	\texttt{make} command.
	
	The \texttt{make install} command will install Guardian into
	\texttt{\${HOME}/bin}, the installation path can be changed with
	\texttt{make install PREFIX=\${PREFIX}}.
}
















