
\section{Generic Calculator}
{

This example demonstrates the more complex file structures the generated
parser is able to handle. This checker expects the given file to follow
the format of a simple arithmetic math expression: addition, subtraction,
multiplication, division and parenthesis. The checker evaluates the expression
and requires that the result be equal to 11.

\begin{lstlisting}[texcl=true, language=MAIA]
%skip: ' ' | '\n';

highest
	: /['0'-'9']+/ #literal
	| '(' add #subexp ')'
	;

multiply
	: highest #base
	| multiply #sub "*" highest #mutme
	| multiply #sub "/" highest #divme
	;

add	: multiply #base
	| add #sub "+" multiply #addme
	| add #sub "-" multiply #subme
	;

start: add #add;

%parse: argv as start;

evalhigh = $highest h -> int:
	h has literal
		? int!(h.literal)
		: evaladd(h.subexp);

evalmult = $multiply m -> int:
	m has base
		? evalhigh(m.base)
		: (m has mutme
			? evalmult(m.sub) * evalhigh(m.mutme)
			: evalmult(m.sub) / evalhigh(m.divme));

evaladd = $add a -> int:
	a has base
		? evalmult(a.base)
		: (a has addme
			? evaladd(a.sub) + evalmult(a.addme)
			: evaladd(a.sub) - evalmult(a.subme));

result = evaladd(start[0].add);

%error: result == 11;
\end{lstlisting}
}










