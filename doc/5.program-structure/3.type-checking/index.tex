
\section{Type Checking and Specialization}
{
	Type checking and specialization happen at the same time and is done
	through one pass.
	
	Type checking checks that the datatype that each expression would return
	matches with the expectations of the larger expression or statement it
	is within. For instance: the ternary expression's conditional subexpression
	must return a boolean value, so there is an \texttt{if} statement that will
	abort Guardian's execution and print a message if that
	is \textit{not} a boolean value.
	Guardian's programming language is statically-typed, meaning
	that the datatype an expression would return can always be correctly
	speculated before execution.
	
	Specializing expressions refers to the process
	of interpreting the expressions
	against the types of their subexpressions, and allocating new expressions
	to represent more accurately the work that needs to be done.
	For instance, Zebu's parse-tree may
	indicate that there is an addition expression, and---through specializing
	its subexpressions---it is determined they would both return
	list values. After checking that the element types of both lists are the
	same, a new list-concatenation expression is created.
	If all subexpressions are literals, meaning that their result is a known
	constant, the expression is evaluated immediately\footnote{Even if
	the subexpressions would result in known constant values, the
	\texttt{!isabspath}, \texttt{!isdir}, \texttt{!isaccessibleto}
	and \texttt{!isexecutableby} special forms will not be evaluated and
	substituted.} and replaced with a literal expression storing the result.
	A function has been written for type-checking and specializing every
	type of expression and statement.
	
	The steps for specializing a let expression mirror the steps for
	evaluating them: the temporary variables are all declared, then sequentially
	specialized, then the body is specialized.
	
	When specializing a variable-use expression, Guardian checks if the
	variable being used was declared using a literal expression. If that
	is the case, the variable usage is replaced with a new literal expression
	of the same value. This effectively eliminates variables to constants.
	
	An expression can refer to a variable before the variables declaration,
	this is very useful for recursive functions, but it also means that
	Guardian has to have knowledge of what variables exist and what
	types they have before it
	specializes their declaration statements. An initial pass over
	the declaration statements
	must be made to determine the types of the variables and to introduce them
	into Guardian's scope before the main
	statement-specialization pass. Some variables's type may be derived from
	grammar types, so the grammar-rule names and their specialized type must
	be introduced into the type-checking scope before either of the two
	mentioned passes.
	
	Statements do specialize, but not much changes. The
	references to zebu's expression trees are replaced with references
	to specialized expressions. Parse statements check that their
	specialized expression for their path results in a string type, error
	otherwise. Assertions check that their specialized expression results
	in a boolean value, error otherwise. If the conditional is a literal
	true boolean value the assertion is removed.
}

































