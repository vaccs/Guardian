
\chapter{Program Structure}
{
	This chapter is provides a more detailed explanation of the
	internal structure and implementation of Guardian.
	In a sense, Chapter 4 answers the question ``how to use it'',
	while this chapter answers the question ``how it works.''
	
	Guardian's \texttt{main()} follows these steps:
	\begin{enumerate}
	{
		\item Parse command-line arguments.
		
		\item Parse input file.
		
		\item Specialize grammar rules.
		
		\item Perform type checking and specialization on statements and expressions.
		
		\item Generate parser and tokenizer.
		
		\item If the \texttt{-m} flag was set, minimize tokenizer.
		
		\item Generate and write source code.
			
		\item Free all data structures.
	}
	\end{enumerate}
	
	\subimport{1.parsing/}{index.tex}
	
	\subimport{2.specializing-grammar-types/}{index.tex}
	
	\subimport{3.type-checking/}{index.tex}
	
	\subimport{4.parser-and-tokenizer-generation/}{index.tex}
	
	\subimport{5.lexer-minimization/}{index.tex}
	
	\subimport{6.source-code-generation/}{index.tex}
}

