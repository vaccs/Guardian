
\section{Specialization of Grammar Types}
{
	For the sake of the type checking phase
	a grammar type needs to be generated for each named grammar-rule pattern.
	Each field in a pattern can either be a token or a reference to
	another grammar rule. The component that the value of a field
	originates from can be parsed multiple times (e.g. \texttt{("a" \#foo)*)}),
	so a field can be set to either capture all of the values for each
	parse (making a list) or just the first.
	The content of a token can be made to be
	automatically converted to an integer, float or
	boolean datatype after it is parsed and before it is assigned to the
	grammar value. If this is not done, the type of a token defaults to a string.
	For each grammar-rule pattern, a new grammar type is created with the
	same name, and fields. Each field's type is derived by the rules above.
	For example:
	
	\begin{lstlisting}[numbers=none, numbers = none, texcl = true, language = MAIA]
A: "foo" /float: ['0'-'9']+/ #bar | "har" #har;
B: "(" (A #a | B #b) ")" | (/"c" | "C"/ #c[])+;
	\end{lstlisting}
	
	The `A' grammar has a `bar` field of float type, and a `har` field of
	string type. The `B' grammar has an `a' field of the type of grammar `A',
	a `b' field of the type of grammar `B', and a `C' field of a
	string list type.
}
