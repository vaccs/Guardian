
\chapter{Command-line Options}
{
	Guardian has several command-line options.
	
	\section{Command-line Options}
	{
		\begin{itemize}
		{
			\item [\texttt{-i <path>}]
			{
				Gives Guardian the path to the initial file containing the
				file specification for which Guardian is to generate a checker.
				If the file specification spans multiple
				files the additional files would need to be
				directly or indirectly \texttt{\%include}-d by the initial file.
			}
			
			\item [\texttt{-o <path>}]
			{
				Gives Guardian the path to where to write the generated source
				code. After Guardian is complete, the generated source code
				would have to be compiled into an executable as a separate step.
				\textbf{Note}: the executable must be
				linked with \texttt{-lgmp} and
				\texttt{-lquadmath}.
			}
			
			\item [\texttt{-m}]
			{
				Tells Guardian to ``minimize'' the generated tokenizer. The
				particular approach Guardian uses to generate its tokenizer's
				state machine can sometimes yield very large transition tables.
				This flag tells Guardian to take another pass over the generated
				state machine, merging states that it can prove to be redundant
				or unneeded. The result is a state machine that functions the
				same way but has as few transitions and states as
				possible.
				This additional pass can sometimes take a long
				time and so it is
				disabled by default.
			}
			
			\item [\texttt{-v}]
			{
				Tells Guardian to be more verbose about the type-checks that
				it performs.
			}
			
			\item [\texttt{-h}]
			{
				Tells Guardian to display information about its command-line
				flags.
			}
		}
		\end{itemize}
	}
}





