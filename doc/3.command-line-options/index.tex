
\chapter{Command-line Options}
{
	Guardian has several command-line options.
	
	\section{Command-line Options}
	{
		\begin{itemize}
		{
			\item [\texttt{-i <path>}]
			{
				Gives Guardian the path to the initial file containing the
				file specification for Guardian to generate a checker for.
				If the file specification spans multiple
				files the additional files would need to be
				directly or indirectly \texttt{\%include}-d by the initial file.
			}
			
			\item [\texttt{-o <path>}]
			{
				Gives Guardian the path to where to write the generated source
				code to. After Guardian is complete, the generated souce code
				would have to be compiled into an executable as a seperate step.
				Remember to link that executable with '\texttt{-lgmp}' and
				'\texttt{-lm}'!
			}
			
			\item [\texttt{-m}]
			{
				Tells Guardian to "minimize" the generated tokenizer. The
				particular appoarch Guardian uses to generate its tokenizer's
				state machine can sometimes yield very large transition tables.
				This flag tells Guardian to run another pass over the generated
				state machine merging states that it can prove to be redundant
				or unneeded. The result is a state machine that functions the
				same way, but has the fewest number of transitions and states.
				This additional pass can sometimes take a long time, so it is
				disabled by default.
			}
			
			\item [\texttt{-v}]
			{
				Tells Guardian to be more \textit{verbose} about the work that it
				is doing to convert the file specification read from the input
				file(s) to C source code.
			}
			
			\item [\texttt{-h}]
			{
				Tells Guardian to print out a usage message similar to the
				content in this section.
			}
		}
		\end{itemize}
	}
}





