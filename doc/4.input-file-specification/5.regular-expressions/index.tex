
\section{Regular Expressions}
\label{sec:regex}
{
	A regular expression is used to describe a set of
	strings quickly and conveniently. The set itself may be infinite.
	Since computers cannot attempt to store the string-sets themselves in
	memory, computers must rely on a kind of ``compressed" form of the set
	to represent
	these potentionally-infinite sets of strings. Determining if a given string
	is in the set from its compressed form requires work and is known
	as ``matching".
	
	For Guardian's application, regular expressions are almost always used
	to describe what would be an acceptable string of characters to read in
	a certain part of the content of a file being parsed.
	
	The operators are discussed in descending operator precedence. Parenthesis
	can be used to raise the precedence of a regular expression.
	
	\subimport{./}{1.character-literals.tex}
	
	\subimport{./}{2.integer-literals.tex}
	
	\subimport{./}{3.string-literals.tex}
	
	\subimport{./}{4.character-sets.tex}
	
	\subimport{./}{5.suffix-operators.tex}
	
	\subimport{./}{6.juxtaposition.tex}
	
	\subimport{./}{7.or-operator.tex}
	
	\subimport{./}{8.examples.tex}
}

