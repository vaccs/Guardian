
\subsection{Slicing Expression}
{
	Slicing is a kind of expression that can be used for creating sublists
	and substrings. The syntax follows the
	form: \texttt{<list-or-string>[<start-index>:<end-index>]}.
	The start index is optional: if missing, the sublist/substring starts at
	the beginning of the given list/string.
	The end index is optional: if missing, the sublist/substring ends at
	the end of the given list/string.
	The start and end index if present must be a subexpression that results
	in an integer value.
	A list getting sliced results in the list value consisting of the elements
	ranging from the start index value up-to-but-not-including the end index.
	A string getting indexed results in a string value consisting of the
	characters ranging from the start index up-to-but-not-including the end
	index.
	
	If Guardian can determine the exact value of given list/string,
	start index and end index, Guardian
	will replace the call with the appropriate list/string construction.
	
	\begin{itemize}
	{
		\item \texttt{[1, 2, 3, 4, 5, 6][1:4]}
		
			Results in the integer list value [2, 3, 4].
		
		\item \texttt{[1, 2, 3, 4, 5, 6][3:6]}
		
			Results in the integer list value [4, 5, 6].
		
		\item \texttt{[1, 2, 3, 4, 5, 6][0:2]}
		
			Results in the integer list value [1, 2].
		
		\item \texttt{[1, 2, 3, 4, 5, 6][3:]}
		
			Results in the integer list value [4, 5, 6].
		
		\item \texttt{[1, 2, 3, 4, 5, 6][:2]}
		
			Results in the integer list value [1, 2].
		
		\item \texttt{"abcdef"[1:4]}
		
			Results in the string value "bcd".
		
		\item \texttt{"abcdef"[3:6]}
		
			Results in the string value "def".
		
		\item \texttt{"abcdef"[0:2]}
		
			Results in the string value "ab".
			
		\item \texttt{"abcdef"[3:]}
		
			Results in the string value "def".
		
		\item \texttt{"abcdef"[:2]}
		
			Results in the string value "ab".
	}
	\end{itemize}
}














