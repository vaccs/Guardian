
\subsection{Indexing Expression}
{
	Indexing is a kind of expression that can be used for accessing elements
	in a value that supports indexing (lists, strings and dictionaries).
	The syntax follows the
	form: \texttt{<list-or-string-or-dict>[<index>]}.
	A list getting indexed results in the value of the element at the given
	index, expecting the expression enclosed in the square brackets to result
	in an integer index value.
	A string getting indexed results in a string one character long of
	the character at that position indicated by the integer value expected
	to result from the square-bracket-enclosed subexpression.
	A dictionary getting indexed performs a lookup on the dictionary, expecting
	the expression enclosed in the square brackets to result in the key value.
	
	If Guardian can determine the exact value of given list/string
	and index, Guardian
	will replace the call with the appropriate list/string construction.
	
	\begin{itemize}
	{
		\item[] \texttt{[1, 2, 3][0]}
		
			Results in the integer value 1.
		
		\item[] \texttt{[1, 2, 3][1]}
		
			Results in the integer value 2.
		
		\item[] \texttt{[1, 2, 3][2]}
		
			Results in the integer value 3.
		
		\item[] \texttt{"abc"[0]}
		
			Results in the string value "a".
		
		\item[] \texttt{"abc"[1]}
		
			Results in the string value "b".
		
		\item[] \texttt{"abc"[2]}
		
			Results in the string value "c".
		
		\item[] \texttt{\{4: 5, 6: 7, 8: 9\}[4]}
		
			Results in the integer value 4.
			
		\item[] \texttt{\{4: 5, 6: 7, 8: 9\}[6]}
		
			Results in the integer value 7.
			
		\item[] \texttt{\{4: 5, 6: 7, 8: 9\}[8]}
		
			Results in the integer value 9.
	}
	\end{itemize}
}





















