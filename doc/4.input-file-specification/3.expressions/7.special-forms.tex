
\subsection{Special Forms}
{
	All special forms have equal operator precedence.
	Special forms are not true functions, they
	cannot be passed like a lambda
	value. A special form is closer to a kind of operator
	that has a function call-like
	syntax.
	
	\subsubsection{\lstinline[language=MAIA, columns=fixed]@all!@ Special Form}
	{
		A call to the \lstinline[language=MAIA, columns=fixed]@all!@ special form is a kind of an expression.
		The special form expects a list of boolean values as its
		single argument and returns a boolean value of true if all the elements
		are true, false otherwise. An empty list of boolean values returns true.
		The intuition there is that there isn't an element that is not true.
		If Guardian can determine all of the boolean values given to a call to an
		\lstinline[language=MAIA, columns=fixed]@all!@ special form, it will
		substitute the call with the
		appropriate \lstinline[language=MAIA, columns=fixed]@true@ or
		\lstinline[language=MAIA, columns=fixed]@false@ expression.
		
		\begin{itemize}
		{
			\item[] \lstinline[language=MAIA, columns=fixed]@all!([true, true, true])@
			
				Results in a true boolean value.
			
			\item[] \lstinline[language=MAIA, columns=fixed]@all!([true, false])@
			
				Results in a false boolean value.
			
			\item[] \lstinline[language=MAIA, columns=fixed]@all!([<bool>])@
			
				Results in a true boolean value.
		}
		\end{itemize}
	}
	
	\subsubsection{\lstinline[language=MAIA, columns=fixed]@any!@ Special Form}
	{
		A call to the \lstinline[language=MAIA, columns=fixed]@any!@ special form is a kind of an expression.
		The special form expects a list of boolean values as its
		single argument and returns a boolean value of true if any the elements
		were true, false otherwise. An empty list of boolean values returns false.
		The intuition there is that there isn't an element that isn't false.
		If Guardian can determine all of the boolean values given to a call to an
		\lstinline[language=MAIA, columns=fixed]@any!@ special form, it will substitute the call with the
		appropriate \lstinline[language=MAIA, columns=fixed]@true@ or \lstinline[language=MAIA, columns=fixed]@false@ expression.
		
		\begin{itemize}
		{
			\item[] \lstinline[language=MAIA, columns=fixed]@any!([true, true, true])@
			
				Results in a true boolean value.
			
			\item[] \lstinline[language=MAIA, columns=fixed]@any!([true, false])@
			
				Results in a true boolean value.
			
			\item[] \lstinline[language=MAIA, columns=fixed]@any!([false])@
			
				Results in a false boolean value.
			
			\item[] \lstinline[language=MAIA, columns=fixed]@any!([<bool>])@
			
				Results in a false boolean value.
		}
		\end{itemize}
	}
	
	\subsubsection{\lstinline[language=MAIA, columns=fixed]@crossmap!@ Special Form}
	{
		A call to the \lstinline[language=MAIA, columns=fixed]@crossmap!@ special form is a kind of expression.
		The special form expects a lambda value as its
		first parameter and expects the remaining parameters to be list values.
		This special form results in the list value consisting
		of all of the return values of the given lambda value called on each of
		the combinations of elements from the given lists.
		\lstinline[language=MAIA, columns=fixed]@crossmap!@ requires at least one list and that
		the element type of each list is equal to the corresponding parameter type
		of the given lambda.
		If Guardian can determine the exact lambda values and lists all call to
		\lstinline[language=MAIA, columns=fixed]@crossmap!@, Guardian will replace the call with a list
		construction with the appropriate elements.
		
		\begin{itemize}
		{
			\item[] \lstinline[language=MAIA, columns=fixed]@crossmap!($int x -> int: x + 3, [1, 2, 3])@
			
				The result of this call to \lstinline[language=MAIA, columns=fixed]@crossmap!@ would be a list of
				integer elements 4, 5 and 6.
			
			\item[] \lstinline[language=MAIA, columns=fixed]@crossmap!($bool x, y -> (bool, bool): (x, y), [true, false], [true, false])@
			
				The result of this call to \lstinline[language=MAIA, columns=fixed]@crossmap!@ would be a list of
				(boolean, boolean)
				tuples: \lstinline[language=MAIA, columns=fixed]@[(true, true), (true, false), (false, true), (false, false)]@.
		}
		\end{itemize}
	}
	
	\subsubsection{\lstinline[language=MAIA, columns=fixed]@filter!@ Special Form}
	{
		A call to the \lstinline[language=MAIA, columns=fixed]@filter!@ special form is a kind of expression.
		The \lstinline[language=MAIA, columns=fixed]@filter!@ special form requires the first parameter to be
		a lambda value and the second parameter to be a list value.
		The lambda value must require one parameter, the parameter's type must
		be the list value's element type, and the lambda value must return a
		boolean type.
		The special form results in a list value consisting of all the elements
		of the given list that resulted in a true when called on the lambda
		value.
		If Guardian can determine the exact lambda value and the exact
		values for the elements in the list, Guardian will replace the call
		to \lstinline[language=MAIA, columns=fixed]@filter!@ with a list construction with the appropriate
		remaining elements.
		
		\begin{itemize}
		{
			\item[] \lstinline[language=MAIA, columns=fixed]@filter!($int x -> bool: true, [1, 2, 3])@
			
				Results in the list of integer values 1, 2 and 3.
			
			\item[] \lstinline[language=MAIA, columns=fixed]@filter!($string s -> bool: "ium" in s, ["hydrogen", "helium", "lithium"])@
			
				Results in the list of string
				values: \lstinline[language=MAIA, columns=fixed]@["helium", "lithium"]@.
			
			\item[] \lstinline[language=MAIA, columns=fixed]@filter!($int[] x -> bool: sum!(x) > 5, [[2, 10, 3], [1, 1, 1], [3, 2, 5])@
				
				Results in the list of lists of
				integers: \lstinline[language=MAIA, columns=fixed]@[[2, 10, 3], [3, 2, 5]]@.
		}
		\end{itemize}
	}
	
	\subsubsection{\lstinline[language=MAIA, columns=fixed]@float!@ Special Form}
	{
		A call to the \lstinline[language=MAIA, columns=fixed]@float!@ special form is a kind of expression.
		The \lstinline[language=MAIA, columns=fixed]@float!@ special form expects
		one parameter and can convert it
		into a float value. Currently only converting from an integer or string
		value is supported.
		If Guardian can determine the exact integer or string value given to
		\lstinline[language=MAIA, columns=fixed]@float!@ Guardian will replace the call with the appropriate
		float literal.
		
		\begin{itemize}
		{
			\item[] \lstinline[language=MAIA, columns=fixed]@float!(2)@
			
				Equalivent to the float value 2.0.
			
			\item[] \lstinline[language=MAIA, columns=fixed]@float!("12")@
			
				Equalivent to the float value 12.0.
			
			\item[] \lstinline[language=MAIA, columns=fixed]@float!("2e3")@
			
				Equalivent to the float value 2000.0.
		}
		\end{itemize}
	}
	
	\subsubsection{\lstinline[language=MAIA, columns=fixed]@isaccessibleto!@ Special Form}
	{
		A call to the \lstinline[language=MAIA, columns=fixed]@isaccessibleto!@ special form
		is a kind of expression.
		The special form expects two string parameters and returns a true boolean
		value if the given path (the first parameter) is accessible (readable) to
		the given user (the second parameter), false
		otherwise.
		Even if Guardian can determine the exact string values the form was
		called with, no substitution is done.
	}
	
	\subsubsection{\lstinline[language=MAIA, columns=fixed]@isdir!@ Special Form}
	{
		A call to the \lstinline[language=MAIA, columns=fixed]@isdir!@ special form is a kind of expression.
		The special form expects one string parameter and returns a true boolean
		value if the given path refers to a directory, false
		otherwise.
		Even if Guardian can determine the exact string value the form was
		called with, no substitution is done.
	}
	
	\subsubsection{\lstinline[language=MAIA, columns=fixed]@isexecutableby!@ Special Form}
	{
		A call to the \lstinline[language=MAIA, columns=fixed]@isexecutableby!@ special form
		is a kind of expression.
		The special form expects two string parameters and returns a true boolean
		value if the given path (the first parameter) is executable by
		the given user (the second parameter), false
		otherwise.
		Even if Guardian can determine the exact string values the form was
		called with, no substitution is done.
	}

	\subsubsection{\lstinline[language=MAIA, columns=fixed]@int!@ Special Form}
	{
		A call to the \lstinline[language=MAIA, columns=fixed]@int!@ special form is a kind of expression.
		The \lstinline[language=MAIA, columns=fixed]@int!@ special form expects one parameter and can convert it
		into an integer value. Currently only converting from a float or string
		value is supported.
		If Guardian can determine the exact float or string value given to
		\lstinline[language=MAIA, columns=fixed]@int!@ Guardian will replace the call with the appropriate
		integer literal.
		
		\begin{itemize}
		{
			\item[] \lstinline[language=MAIA, columns=fixed]@int!(2.0)@
			
				Equalivent to the integer value 2.
			
			\item[] \lstinline[language=MAIA, columns=fixed]@int!("12")@
			
				Equalivent to the integer value 12.
			
			\item[] \lstinline[language=MAIA, columns=fixed]@int!(3.5)@
			
				Equalivent to the integer value 3.
		}
		\end{itemize}
	}
	
	\subsubsection{\lstinline[language=MAIA, columns=fixed]@isabspath!@ Special Form}
	{
		A call to the \lstinline[language=MAIA, columns=fixed]@isabspath!@ special form is a kind of expression.
		The special form expects one string parameter and returns a true boolean
		value if the given string matches the form of an absolute path, false
		otherwise.
		Even if Guardian can determine the exact string value the form was
		called with, no substitution is done.
		
		\begin{itemize}
		{
			\item[] \lstinline[language=MAIA, columns=fixed]@isabspath!("/abc/dev/")@
			
				On a unix filesystem, the form would return true.
			
			\item[] \lstinline[language=MAIA, columns=fixed]@isabspath!("../abc/dev")@
			
				On a unix filesystem, the form would return false.
		}
		\end{itemize}
	}
	
	\subsubsection{\lstinline[language=MAIA, columns=fixed]@len!@ Special Form}
	{
		A call to the \lstinline[language=MAIA, columns=fixed]@len!@ special form is a kind of expression.
		The \lstinline[language=MAIA, columns=fixed]@len!@ special form will return an integer value signifying
		the length of the given value. Currently string, tuple, set
		and dictionary values are supported.
		If Guardian can determine the exact string, tuple, set or dictionary
		value the form is being called with, Guardian will replace the call with
		the appropriate integer literal.
		
		\begin{itemize}
		{
			\item[] \lstinline[language=MAIA, columns=fixed]@len!((1, true, 3.0))@
			
				Equalivent to the integer value 3.
			
			\item[] \lstinline[language=MAIA, columns=fixed]@len!([<bool>])@
			
				Equalivent to the integer value 0.
			
			\item[] \lstinline[language=MAIA, columns=fixed]@len!({1, 2, 3} | {2, 3, 4})@
			
				Equalivent to the integer value 4.
		}
		\end{itemize}
	}
	
	\subsubsection{\lstinline[language=MAIA, columns=fixed]@map!@ Special Form}
	{
		A call to the \lstinline[language=MAIA, columns=fixed]@map!@ special form is a kind of expression.
		The special form expects a lambda value as its
		first parameter and expects the remaining parameters to be list values.
		This special form results in the list value consisting
		of all of the return values of the given lambda value called on all of the
		first elements from the given lists, second elements from the given
		lists, up to the length of the shortest list.
		\lstinline[language=MAIA, columns=fixed]@map!@ requires at least one list and that
		the element type of each list is equal to the corresponding parameter type
		of the given lambda.
		If Guardian can determine the exact lambda values and lists all call to
		\lstinline[language=MAIA, columns=fixed]@map!@, Guardian will replace the call with a list
		construction with the appropriate elements.
		
		\begin{itemize}
		{
			\item[] \lstinline[language=MAIA, columns=fixed]@map!($int x -> int: x + 3, [1, 2, 3])@
			
				The result of this call to \lstinline[language=MAIA, columns=fixed]@map!@ would be a list of
				integer elements 4, 5 and 6.
			
			\item[] \lstinline[language=MAIA, columns=fixed]@map!($bool x, y -> (bool, bool): (x, y), [true, false], [true, false])@
			
				The result of this call to \lstinline[language=MAIA, columns=fixed]@map!@ would be list of
				(boolean, boolean) tuples: \lstinline[language=MAIA, columns=fixed]@[(true, true), (false, false)]@.
		}
		\end{itemize}
	}
	
	\subsubsection{\lstinline[language=MAIA, columns=fixed]@range!@ Special Form}
	{
		A call to the \lstinline[language=MAIA, columns=fixed]@range!@ special form is a kind of expression.
		The special form expects one or two integer value parameters.
		The form results in a list of integer values up-to-but-not-including
		the last value, starting from the other value or 0 if called with only
		one value.
		If Guardian can determine the exact values a call to \lstinline[language=MAIA, columns=fixed]@range!@
		was made with, it will substitute the call with a list construction
		with the appropriate elements.
		
		\begin{itemize}
		{
			\item[] \lstinline[language=MAIA, columns=fixed]@range!(10)@
			
				Results in \lstinline[language=MAIA, columns=fixed]@[0, 1, 2, 3, 4, 5, 6, 7, 8, 9]@.
			
			\item[] \lstinline[language=MAIA, columns=fixed]@range!(2, 6)@
			
				Results in \lstinline[language=MAIA, columns=fixed]@[2, 3, 4, 5]@.
				
			\item[] \lstinline[language=MAIA, columns=fixed]@range!(-3, 3)@
			
				Results in \lstinline[language=MAIA, columns=fixed]@[-3, -2, -1, 0, 1, 2]@.
		}
		\end{itemize}
	}
	
	\subsubsection{\lstinline[language=MAIA, columns=fixed]@reduce!@ Special Form}
	{
		A call to the \lstinline[language=MAIA, columns=fixed]@reduce!@ special form is a kind of expression.
		The special form expects three parameters.
		The first parameter must be a lambda value, and the
		second parameter must be a list value.
		The lambda value's first parameter's type
		must match the element type of the list value and the type of the third
		parameter. The lambda value's
		second parameter's type must match the lambda value's return type.
		
		The lambda value is called repeatedly, once for each element of the list.
		The lambda is called with the result of the last time it was
		called (initially the third parameter), and
		the current element.
		
		If Guardian can determine the exact lambda and list value used in a call to
		\lstinline[language=MAIA, columns=fixed]@reduce!@, Guardian will replace the call with a literal of
		the appropriate value.
		
		\begin{itemize}
		{
			\item[] \lstinline[language=MAIA, columns=fixed]@reduce!($int x, y -> int: x + y, [1, 2, 3], 0)@
			
				First the lambda is called with 0 and 1 yielding 1.
				Then the lambda is called with 1 and 2 yielding 3.
				Then the lambda is called with 3 and 3 yielding 6.
			
			\item[] \lstinline[language=MAIA, columns=fixed]@reduce!($int x, y -> int: x > y ? x : y, [2, 3, 1, 4], 0)@
			
				First the lambda is called with 0 and 2 yielding 2.
				Then the lambda is called with 2 and 3 yielding 3.
				Then the lambda is called with 3 and 1 yielding 3.
				Then the lambda is called with 3 and 4 yielding 4.
			
			\item[] \lstinline[language=MAIA, columns=fixed]@reduce!($int x, y -> int: x < y ? x : y, [2, 3, 1, 4], 5)@
			
				First the lambda is called with 5 and 2 yielding 2.
				Then the lambda is called with 2 and 3 yielding 2.
				Then the lambda is called with 2 and 1 yielding 1.
				Then the lambda is called with 1 and 4 yielding 1.
				
			\item[] \lstinline[language=MAIA, columns=fixed]@reduce!($int x, int[] y -> int[]: x \% 2 == 0 ? y + [x] : y, [2, 3, 1, 4], [<int>]@
			
				First the lambda is called with \lstinline[language=MAIA, columns=fixed]@[<int>]@ and 2 yielding \lstinline[language=MAIA, columns=fixed]@[2]@.
				Then the lambda is called with \lstinline[language=MAIA, columns=fixed]@[2]@ and 3 yielding \lstinline[language=MAIA, columns=fixed]@[2]@.
				Then the lambda is called with \lstinline[language=MAIA, columns=fixed]@[2]@ and 1 yielding \lstinline[language=MAIA, columns=fixed]@[2]@.
				Then the lambda is called with \lstinline[language=MAIA, columns=fixed]@[2]@ and 4 yielding \lstinline[language=MAIA, columns=fixed]@[2, 4]@.
				
			\item[] \lstinline[language=MAIA, columns=fixed]@reduce!($int x, y -> int: 1, [<int>], 3)@
			
				Due to the list consisting of no elements, the third parameter
				is returned.
		}
		\end{itemize}
	}
	
	\subsubsection{\lstinline[language=MAIA, columns=fixed]@sum!@ Special Form}
	{
		A call to the \lstinline[language=MAIA, columns=fixed]@sum!@ special form is a kind of expression.
		The special form expects one single parameter, either a list of integer
		or float values. An empty integer list results in 0. An empty float list
		results in a 0.0.
		If Guardian can determine the exact values in the given list, Guardian
		will replace the call with the appropriate integer or float value.
		
		\begin{itemize}
		{
			\item[] \lstinline[language=MAIA, columns=fixed]@sum!([1, 2, 3])@
			
				Equalivent to the integer value 6.
			
			\item[] \lstinline[language=MAIA, columns=fixed]@sum!([1.0, 2.0, 3.0])@
			
				Equalivent to the float value 6.0
			
			\item[] \lstinline[language=MAIA, columns=fixed]@sum!([<int>])@
			
				Equalivent to the integer value 0.
		}
		\end{itemize}
	}
}






















