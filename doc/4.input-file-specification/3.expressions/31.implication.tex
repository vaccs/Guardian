
\subsection{Implication Expression}
{
	Implication is a kind of expression that expects two subexpressions that
	both yield a boolean value. The expression results in a true boolean value
	if the first subexpression resulted in a false, otherwise it returns
	the result of the second subexpression.
	
	The implication expression supports short-circuiting: if the first
	subexpression produces false, the implication expression returns true
	without evaluating the second expression.
	
	If Guardian can determine the exact value of the two given subexpressions,
	Guardian will replace the call with the appropriate value literal.
	
	\begin{itemize}
	{
		\item[] \texttt{true implies true}
		
			Results in the true boolean value.
			
		\item[] \texttt{true implies false}
		
			Results in the false boolean value.
			
		\item[] \texttt{false implies true}
		
			Results in the true boolean value.
			
		\item[] \texttt{false implies false}
		
			Results in the true boolean value.
	}
	\end{itemize}
}
