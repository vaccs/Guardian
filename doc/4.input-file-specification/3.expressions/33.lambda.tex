
\subsection{Lambda Expression}
{
	Use of the \texttt{lambda} keyword is a kind of expression, resulting
	in a new lambda value. A lambda value references a kind of
	one-expression function that accepts parameters and produces a return value.
	The syntax for a lambda expression follows the
	form: \texttt{\$<0th-parameter-type> <0th-parameter-name>,
	<1th-parameter-type> <1th-parameter-name>, ... <nth-parameter-type>
	<nth-parameter-name> -> <return-type>: <body>}.
	After the first parameter, all parameter types are optional; the
	parameters that do not give their type explicitly will be assumed to
	have the same type as the last explicit parameter type.
	
	Lambda body expressions can refer to variables in the larger scope of their
	declarations.
	
	\begin{itemize}
	{
		\item \texttt{\$int x -> int: x + 3}
		
			Results in a lambda value that expects
			one integer parameter and, when called, produces integer sum
			of its parameter and 3.
		
		\item \texttt{\$int x, y, z -> int: x + y + z}
		
			Results in a lambda value that expects
			three integer parameters and, when called, produces the integer sum
			of its three parameters.
		
		\item \texttt{\$(\$int -> int) f, int x -> (\$ -> int): \$->int: f(x)}
		
			Results in a lambda that expects two parameters:
			a lambda that expects an integer and produces an integer,
			and an integer. The lambda results in a lambda that returns the
			result of calling the given lambda on the given integer.
		
		\item \texttt{\$int x -> (\$int -> (\$->int)): \$int y -> (\$->int): \$->int: x + y}
		
			Results in a lambda that expects an integer parameter, and returns
			a lambda than expects another integer parameter, and returns
			a lambda that expects no parameters and returns the integer sum
			of the two parameters.
	}
	\end{itemize}
}














