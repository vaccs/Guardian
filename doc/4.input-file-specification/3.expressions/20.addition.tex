
\subsection{Addition and Subtraction Expressions}
{
	Addition and subtraction are kinds of expressions that are typically
	add and subtract numeric values (integer and float values).
	The addition operator can also be used to concatenate string values,
	merge dictionary values, union set values and concatenate tuple values.
	
	If Guardian can determine the exact value of the two given subexpressions,
	Guardian will replace the call with the appropriate value literal.
	
	\begin{itemize}
	{
		\item[] \lstinline[language=MAIA, columns=fixed]@2 + 3@
		
			Results in the integer value 5.
		
		\item[] \lstinline[language=MAIA, columns=fixed]@3.5 + 2.5@
		
			Results in the float value 6.0.
		
		\item[] \lstinline[language=MAIA, columns=fixed]@3 - 2@
		
			Results in the integer value 1.
		
		\item[] \lstinline[language=MAIA, columns=fixed]@3.5 - 2.5@
		
			Results in the float value 1.0.
		
		\item[] \lstinline[language=MAIA, columns=fixed]@"abc" + "def"@
		
			Results in the string value "abcdef".
			
		\item[] \lstinline[language=MAIA, columns=fixed]@{1: 2, 2: 3} + {2: 4, 3: 5}@
		
			Results in the integer-to-integer
			dictionary value \lstinline[language=MAIA, columns=fixed]@{1: 2, 2: 4, 3: 5}@.
			
		\item[] \lstinline[language=MAIA, columns=fixed]@{1, 2} + {3, 2}@
		
			Results in the integer set value \lstinline[language=MAIA, columns=fixed]@{1, 2, 3}@.
			
		\item[] \lstinline[language=MAIA, columns=fixed]@(1, 2) + (2.0, 3.0)@
		
			Results in the integer-integer-float-float
			tuple value \lstinline[language=MAIA, columns=fixed]@(1, 2, 2.0, 3.0)@.
			
	}
	\end{itemize}
}
















