
\subsection{Bitwise-xor Expression}
{
	Bitwise-xor is a kind of expression that typically performs
	a bitwise-xor operation on two given integer values.
	The expression can also be used to take the set-symmetric-difference
	of two given
	set values, resulting in a set of the elements that are contained in
	\textbf{exactly} one of the two given sets.
	The expression can also be used to take the
	symmetric-difference of two given
	dictionary values, resulting in a dictionary of all of the key-value
	associations contained in \textbf{exactly} one of the two given
	dictionaries.
	
	If Guardian can determine the exact value of the two given subexpressions,
	Guardian will replace the call with the appropriate value literal.
	
	\begin{itemize}
	{
		\item[] \lstinline[language=MAIA, columns=fixed]@6 ^ 3@
		
			Results in the integer value 5.
		
		\item[] \lstinline[language=MAIA, columns=fixed]@{1, 2, 3, 6} ^ {5, 2, 4, 3}@
		
			Results in the integer set value \{1, 4, 5, 6\}.
		
		\item[] \lstinline[language=MAIA, columns=fixed]@{1: 1, 2: 2, 3: 3, 6: 8} ^ {5: 4, 2: 5, 4: 6, 3: 7}@
		
			Results in the integer-to-integer dictionary
			value \lstinline[language=MAIA, columns=fixed]@{1: 1, 2: 5, 3: 7, 6: 8}@.
	}
	\end{itemize}
}
