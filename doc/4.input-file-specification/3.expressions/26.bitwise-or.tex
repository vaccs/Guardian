
\subsection{Bitwise-or Expression}
{
	Bitwise-or is a kind of expression that typically performs
	a bitwise-or operation on two given integer values.
	The expression can also be used to take the set-union
	of two given set values, resulting in a set of the
	elements that are contained in
	\textbf{either} one of the two given sets.
	The expression can also be used to take the
	dictionary-union of two given
	dictionary values, resulting in a dictionary of
	all of the key-value
	associations that \textbf{either} one of the two given
	dictionaries contain. If there is a value in both dictionaries
	associated with the same key, the value of the second dictionary is used.
	
	If Guardian can determine the exact value of the two given subexpressions,
	Guardian will replace the call with the appropriate value literal.
	
	\begin{itemize}
	{
		\item[] \lstinline[language=MAIA, columns=fixed]@6 | 3@
		
			Results in the integer value 7.
		
		\item[] \lstinline[language=MAIA, columns=fixed]@{1, 2, 3, 6} | {5, 2, 4, 3}@
		
			Results in the integer set value \lstinline[language=MAIA, columns=fixed]@{1, 2, 3, 4, 5, 6}@.
		
		\item[] \lstinline[language=MAIA, columns=fixed]@{1: 1, 2: 2, 3: 3, 6: 8} | {5: 4, 2: 5, 4: 6, 3: 7}@
		
			Results in the integer-to-integer dictionary
			value \lstinline[language=MAIA, columns=fixed]@{1: 1, 2: 5, 3: 7, 4: 6, 6: 8}@.
	}
	\end{itemize}
}
