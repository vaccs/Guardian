
\subsection{Inclusion Expression}
{
	Inclusion is a kind of expression that results in a boolean value of
	the result of the first expression "is inside" of the result of the
	second expression. Strings, lists, sets and dictionary types are currently
	supported.
	When the \texttt{in} operator is done
	on two strings the boolean value returned indicates if the first string
	is a substring of the second.
	When the \texttt{in} operator is done on a value and a list/set
	the boolean value returned indicates if the first value
	is an element of the list/set.
	When the \texttt{in} operator is done on a value and a \texttt{dict}
	the boolean value returned indicates if the first value
	is a key of the \texttt{dict}.
	
	If Guardian can determine the exact value of the two given subexpressions,
	Guardian will replace the call with the appropriate boolean literal.
	
	\begin{itemize}
	{
		\item \texttt{"ab" in "abc"}
		
			Results in a true boolean value.
		
		\item \texttt{"ba" in "abc"}
		
			Results in a false boolean value.
		
		\item \texttt{3 in \{1, 3, 2, 5\}}
		
			Results in a true boolean value
		
		\item \texttt{4 in \{1, 3, 2, 5\}}
		
			Results in a false boolean value
		
		\item \texttt{3 in \{1: 10, 3: 30, 2: 20, 5: 50\}}
		
			Results in a true boolean value
		
		\item \texttt{4 in \{1: 10, 3: 30, 2: 20, 5: 50\}}
		
			Results in a false boolean value
	}
	\end{itemize}
}
