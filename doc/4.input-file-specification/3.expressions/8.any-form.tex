
\subsection{\texttt{any!} Special Form}
{
	A call to the \texttt{any!} special form is a kind of an expression.
	The special form expects a list of boolean values as its
	single argument and returns a boolean value of true if any the elements
	were true, false otherwise. An empty list of boolean values returns false.
	The intuition there is that there isn't any element that isn't false.
	If Guardian can determine all of the boolean values given to a call to an
	\texttt{any!} special form, it will substitute the call with the
	appropriate \texttt{true} or \texttt{false} expression.
	
	\begin{itemize}
	{
		\item \texttt{any!([true, true, true])}
		
			Results in a true boolean value.
		
		\item \texttt{any!([true, false])}
		
			Results in a true boolean value.
		
		\item \texttt{any!([false])}
		
			Results in a false boolean value.
		
		\item \texttt{any!([<bool>])}
		
			Results in a false boolean value.
	}
	\end{itemize}
}
