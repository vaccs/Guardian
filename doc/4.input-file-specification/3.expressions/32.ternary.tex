
\subsection{Ternary Expressions}
{
	C-style ternary expressions are supported.
	They serve the purpose of being an inline \texttt{if}
	statement, returning the result of the true-case expression
	if the given conditional evaluates to true, and returning the result
	of the false-case expression otherwise.
	Both the true-case and false-case expressions must return the same
	datatype. The conditional expression must return a boolean type.
	The syntax for a ternary expression follows the
	form: \texttt{<conditional> ? <true-case> : <false-case>}.
	
	\begin{itemize}
	{
		\item \texttt{true ? 3 : 4}
		
			Results in the integer value 3.
		
		\item \texttt{false ? 3 : 4}
		
			Results in the integer value 4.
	}
	\end{itemize}
}
