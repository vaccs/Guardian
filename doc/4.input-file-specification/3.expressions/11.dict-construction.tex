
\subsection{Dict-Construction Expression}
{
	Dict construction is a kind of expression and can be
	used to create and initialize new dict values.
	The values for each element must be given has a curly-bracket-enclosed
	series of comma-delimited
	key-value pairs of
	subexpressions: \texttt{\{<0th-key>: <0th-value>, <1th-key>: <1th-value>, ... <nth-key>: <nth-value>\})}
	The syntax for an empty dict value must include the key and value
	type enclosed
	in angle-brackets: \texttt{\{<<key-type>: <value-type>>\}}.
	If there are duplicate keys in a dict construction, the
	association occurring latest in
	the dict will be kept.
	
	\begin{itemize}
	{
		\item[] \texttt{\{1: false, 2: false, 1: false, 3: true, 1: true\}}
		
			A integer-to-integer dict value of 1, 2 and 3, mapping to true,
			false and true, respectively.
		
		\item[] \texttt{\{<float: int>\}}
		
			An empty float-to-int dict value.
		
		\item[] \texttt{\{true: false\}}
		
			A boolean-to-boolean dict value of
			true mapping false.
		
		\item[] \texttt{\{<(int, float): (float, int)>\}}
		
			An empty dict mapping
			tuples of integers and floats to tuples of floats to integers.
	}
	\end{itemize}
}
