
\section{Expressions}
{
	Expressions are the way Guardian operates on data and is where the power
	of the programming language lies.
	
	Guardian's programming language is functional, meaning that instead of
	loops and conditional jumps controlling
	execution, programs are
	instead constructed by applying and composing functions. Function
	definitions are trees of expressions that map values to other values,
	rather than a sequence of imperative statements which update
	the running state of the program. One way to think of it is that
	a functional program runs similar to how the way a math equation is
	calculated: different operators have rules about which ones run before
	other ones, and data dependency---not the line number---determines which
	operations execute next.
	
	Syntax and expressions are discussed in order of decreasing operator
	precedence. Remember that parentheses can be used to raise operator
	precedence. All Special forms have equal precedence.
	It is not expected that the forthcoming sections will be read
	in order. Rather, sections of interest or of need can be jumped to first
	and less critical sections (but still ones that are good to know) can be
	returned to later.
	
	\subimport{./}{1.true.tex}
	\subimport{./}{2.false.tex}
	\subimport{./}{3.integer-literal.tex}
	\subimport{./}{4.float-literal.tex}
	\subimport{./}{5.string-literal.tex}
	\subimport{./}{6.identifier.tex}
	\subimport{./}{7.special-forms.tex}
	\subimport{./}{8.tuple-construction.tex}
	\subimport{./}{9.list-construction.tex}
	\subimport{./}{10.set-construction.tex}
	\subimport{./}{11.dict-construction.tex}
	\subimport{./}{12.indexing.tex}
	\subimport{./}{13.slicing.tex}
	\subimport{./}{14.tuple-indexing.tex}
	\subimport{./}{15.field-accessing.tex}
	\subimport{./}{16.function-call.tex}
	\subimport{./}{17.unary-operators.tex}
	\subimport{./}{18.exponentiation.tex}
	\subimport{./}{19.multiplication.tex}
	\subimport{./}{20.addition.tex}
	\subimport{./}{21.bitshift-operations.tex}
	\subimport{./}{22.comparison.tex}
	\subimport{./}{23.matching.tex}
	\subimport{./}{24.bitwise-and.tex}
	\subimport{./}{25.bitwise-xor.tex}
	\subimport{./}{26.bitwise-or.tex}
	\subimport{./}{27.logical-and.tex}
	\subimport{./}{28.logical-or.tex}
	\subimport{./}{29.inclusion.tex}
	\subimport{./}{30.possession.tex}
	\subimport{./}{31.implication.tex}
	\subimport{./}{32.ternary.tex}
	\subimport{./}{33.lambda.tex}
	\subimport{./}{34.let.tex}
}











