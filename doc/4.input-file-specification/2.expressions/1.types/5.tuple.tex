
\subsubsection{Tuple}
{
	The tuple type is used to store
	multiple values of \textbf{mixed} data type,
	it functions as a generic method for packaging multiple values
	as one (Think \texttt{struct}s from C).
	Tuple types are specialized to the number of elements and their element's
	types. The syntax for referring to a tuple type
	is \texttt{(<0th-elements-type>, <1th-elements-type>, ... <nth-elements-type>)}.
	
	The syntax for a tuple value is to comma separate the elements/fields
	of the tuple, enclosed in
	parentheses: \texttt{(<0th-element>, <1th-element>, ... <nth-element>)}.
	
	To avoid confusion between normal parentheses
	and a tuple of one field, a tuple of one field must have a extra comma after
	the element: \texttt{(<element>, )}.
	
	The syntax for an empty tuple is: \texttt{()}.
	
	Tuples can be indexed using the
	syntax: \texttt{<list>.<index>}. The
	index must be an integer constant.
	
	Tuples can be concatenated to each other using the '\texttt{+}' operator.
	
	A tuple is comparable only if all of its elements are comparable.
	Lexicographical equal-to, not-equal-to, greater-than, greater-than-equal-to,
	less-than, less-than-equal-to comparisons can be done on tuple values with
	the '\texttt{==}', '\texttt{!=}',
	'\texttt{>}', '\texttt{>=}', '\texttt{<}', '\texttt{<=}' respectively.
}
