
\section{Expressions}
{
	Expressions are the way Guardian operates on data and is where the meat
	of the programming language lies.
	
	Guardian's programming language is functional, meaning that instead of
	for loops and if statements etc. controling execution, programs are
	instead constructed by applying and composing functions. Function
	definitions are trees of expressions that map values to other values,
	rather than a sequence of imperative statements which update
	the running state of the program. One way to think of it is that
	a functional program runs like how a math equation is
	calculated: different operators have rules about which ones run before
	other ones, and where data dependency has a stronger control over which
	operations execute next then the line number.
	
	Guardian's programming language is statically-typed, meaning that the
	datatypes
	must be known at the time Guardian reads the language and is
	converting it to C.
	
	\subimport{./1.types/}{index.tex}
	
	\subimport{./2.literals/}{index.tex}
	
	\subimport{./3.variables/}{index.tex}
	
	\subimport{./4.boolean-operations/}{index.tex}
	
	\subimport{./5.integer-operations/}{index.tex}
	
	\subimport{./6.float-operations/}{index.tex}
	
	\subimport{./7.string-operations/}{index.tex}
	
	\subimport{./8.list-operations/}{index.tex}
	
	\subimport{./9.tuple-operations/}{index.tex}
	
	\subimport{./10.set-operations/}{index.tex}
	
	\subimport{./11.dict-operations/}{index.tex}
	
	\subimport{./12.comparision-operators/}{index.tex}
	
	\subimport{./13.has-operators/}{index.tex}
	
	\subimport{./14.in-operators/}{index.tex}
	
	\subimport{./15.regex-matches/}{index.tex}
	
	\subimport{./16.ternarys/}{index.tex}
	
	\subimport{./17.parentheses/}{index.tex}
	
	\subimport{./18.field-accesses/}{index.tex}
	
	\subimport{./19.function-calls/}{index.tex}
	
	\subimport{./20.lambdas/}{index.tex}
	
	\subimport{./21.lets/}{index.tex}
	
	\subimport{./22.special-forms/}{index.tex}
	
}

