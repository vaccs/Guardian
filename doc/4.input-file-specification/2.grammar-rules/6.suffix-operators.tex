
\subsection{Repetition Operators}
{
	Guardian supports several operators for indicating
	repetition for grammar-rule patterns:
	
	\begin{itemize}
	{
		\item[\texttt{(...)?}] Indicates that the given subpattern is optional
			and may occur zero or once.
			
			Example: \texttt{"a"?} would match \texttt{""} or \texttt{"a"}.
		\item[\texttt{(...)+}] Indicates that the given subpattern may occur
			more than once.
			
			Example: \texttt{"a"+} would match \texttt{"a"},
				\texttt{"aa"}, \texttt{"aaa"}, etc..
		\item[\texttt{(...)*}] Indicates that the given subpattern may occur
			zero or more times.
			
			Example: \texttt{"a"+} would match \texttt{""}, \texttt{"a"},
				\texttt{"aa"}, \texttt{"aaa"}, etc..
		\item[\texttt{(...)\{n\}}] Indicates that the given subpattern will
			repeat \textit{exactly} \texttt{n} number of times.
			\texttt{n} must be an integer
			literal.
			
			Example: \texttt{"a"\{3\}} matches \texttt{"aaa"}.
		\item[\texttt{(...)\{n,\}}] Indicates that the given subpattern will
			repeat at \textit{least} \texttt{n} number of
			times. \texttt{n} must be an integer
			literal.
			
			Example: \texttt{"a"\{3,\}} matches \texttt{"aaa"}, \texttt{"aaaa"},
			\texttt{"aaaaa"}, etc.
		\item[\texttt{(...)\{,m\}}] Indicates that the given subpattern will
			repeat at \textit{most} \texttt{m} number of
			times. \texttt{m} must be an integer literal.
			
			Example: \texttt{"a"\{,3\}} matches \texttt{""}, \texttt{"a"},
			\texttt{"aa"}, \texttt{"aaa"}.
		\item[\texttt{(...)\{n,m\}}] Indicates that the given subpattern will
			repeat \textit{between} \texttt{n} and \texttt{m} number of
			times. \texttt{n} and \texttt{m} must be integer literals.
			
			Example: \texttt{"a"\{2,4\}} matches \texttt{"aa"}, \texttt{"aaa"},
			and \texttt{"aaaa"}.
	}
	\end{itemize}
}



















