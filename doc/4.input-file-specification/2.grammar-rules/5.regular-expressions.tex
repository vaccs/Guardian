
\subsection{Regular Expressions}
{
	Regular expressions can be used in a grammar-rule pattern.
	The regular expression describes the set of strings that the token
	that should be
	expected
	to be read at this point in the the grammar-rule pattern can consist of.
	
	The syntax for using a regular expression to describe the desired token
	in the grammar-rule pattern is: \texttt{/<regular-expression>/}.
	
	This component of a grammar-rule pattern can be named and would by default
	appear to
	the programming language as a string consisting of the characters matched
	by the given regular expression. This can be changed to the string
	instead being
	parsed as a number and appearing
	as an integer or as a float to the programming language
	by using the syntax:
	\texttt{/int: <regular-expression>/} or
	\texttt{/float: <regular-expression>/} respectively.
	The string could also be converted to a boolean value (either the string must
	be exactly \texttt{"true"} or \texttt{"false"}, anything else is an error)
	with the \texttt{/bool: <regular-expression>/} syntax.
	
	The exact syntax of regular expressions is covered in
	\hyperref[sec:regex]{its own section}.
	
}
