
\subsection{List Type}
{
	The list type is used to store multiple values of the same data type,
	it functions as arrays from other languages.
	List types are specialized to the type of their element.
	The syntax for referring to a list type is: \texttt{<element-type>[]}.
	
	The syntax for a list value is to comma separate the list elements
	and to enclose with square
	brackets: \texttt{[<0th-element>, <1th-element>, ... <nth-element>]}.
	
	The syntax for an empty list value is: \texttt{[<<element-type>>]}
	
	Lists can be indexed using the
	syntax: \texttt{<list>[<index>]}. The
	index must be an integer.
	Lists can be "sliced" (getting a sublist) using
	the syntax: \texttt{<list>[<start-index>:<end-index>]}. Both
	indexes must be integers. The returned list would consist of
	the elements starting from the start index up to but not including
	the end index.
	
	Lists can be concatenated to each other using the \texttt{+} operator.
	
	Lexicographical equal-to, not-equal-to, greater-than, greater-than-equal-to,
	less-than, and less-than-equal-to comparisons
	can be done on list values with
	the \texttt{==}, \texttt{!=},
	\texttt{>}, \texttt{>=}, \texttt{<}, and \texttt{<=} respectively.
	
	The length of a list value can be found with the \texttt{len!} special form.
}




















