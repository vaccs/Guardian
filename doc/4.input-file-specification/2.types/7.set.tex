
\subsection{Set Type}
{
	The set type is used to store
	multiple values of the same data type with no
	particular order.
	Set types are specialized to the type of their elements.
	The syntax for referring to a set type
	is \texttt{<element-type>\{\}}.
	
	The syntax for a set value is to comma separate the elements/fields
	of the tuple, enclosed in
	curly-brackets: \texttt{\{<0th-element>, <1th-element>, ... <nth-element>\}}.
	
	The syntax for an empty set value is: \texttt{\{<<element-type>>\}}.
	
	Set values cannot indexed.
	Set values support the union operation from set-mathematics using
	either the \texttt{|} or \texttt{+} operator.
	Difference can be done with the \texttt{-} operator.
	Intersection can be done with the \texttt{\&} operator.
	Symmetric difference can be done with the \texttt{\^} operator.
	
	Any type that is \textit{comparable} can be the element type of a set.
	(Currently booleans, integers, floats, strings, lists, tuples,
		sets, and dictionaries. Not lambdas nor grammar types)
	
	Lexicographical equal-to, not-equal-to, greater-than, greater-than-equal-to,
	less-than, and less-than-equal-to comparisons
	can be done on set values with
	the \texttt{==}, \texttt{!=},
	\texttt{>}, \texttt{>=}, \texttt{<}, and \texttt{<=} respectively.
	
	The length of a list value can be found with the \texttt{len!} special form.
}
