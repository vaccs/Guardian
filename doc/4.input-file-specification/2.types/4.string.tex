
\subsection{String}
{
	The \texttt{string} type is used to store strings of
	text (sequential characters).
	Each character is stored as a byte, and the length is explicitly
	maintained. There is no null terminator.
	The syntax for referring to a string type is: \texttt{string}.
	
	The syntax for a string value follows the rules for C-style strings.
	The tab, newline, double-quote, single-quote and slash escape sequences
	are supported.
	
	Strings can be indexed using the syntax: \texttt{<string>[<index>]}. The
	index must be an integer.
	Strings can be "sliced" (getting a substring) using
	the syntax: \texttt{<string>[<start-index>:<end-index>]}. Both
	indexes must be integers. The returned substring would consist of
	the characters starting from the start index up to but not including
	the end index.
	
	Strings can be concatenated to each other using the '\texttt{+}' operator.
	
	Alphabetical equal-to, not-equal-to, greater-than, greater-than-equal-to,
	less-than, less-than-equal-to comparisons can be done on string values with
	the '\texttt{==}', '\texttt{!=}',
	'\texttt{>}', '\texttt{>=}', '\texttt{<}', '\texttt{<=}' respectively.
	
	Strings can be converted into
	an \texttt{int} or a \texttt{float} using
	the \texttt{int!} and \texttt{float!} special form respectively.
}
