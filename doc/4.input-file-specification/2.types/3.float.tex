
\subsection{Float type}
{
	The \texttt{float} type is used to store decimal (real) numbers.
	Guardian and the checkers it generates internally
	use IEEE-754 floating point
	128-bit values (GCC's \texttt{\_\_float128} type) to represent the number.
	15 bits are used for the exponent, 113 bits are used for the mantissa.
	The syntax for referring to the float type is: \texttt{float}.
	
	The syntax for a float value follows the syntax for C-style float literals.
	Normal and exponential notation are supported.
	
	Float values can be added (\texttt{+}),
	subtracted (\texttt{-}), multiplied (\texttt{*}), and
	divided (\texttt{/}) into each
	other. Exponentiation is also supported
	with the \texttt{**} operator.
	
	Equal-to, not-equal-to, greater-than, greater-than-equal-to,
	less-than, and less-than-equal-to comparisons can
	be done on float values with
	the \texttt{==}, \texttt{!=},
	\texttt{>}, \texttt{>=}, \texttt{<}, and \texttt{<=} respectively.
	
	Floats can be converted into integers using
	the \texttt{int!}
	special form.
}
