
\subsection{int}
{
	The \texttt{int} type (short for integer) is used to store integer values.
	Integer values have no theoretical upper or lower limit, as Guardian will
	automatically allocate more memory for representing larger numbers.
	Guardian and the checkers it generates internally
	use mpz integers from the GMP library to make this
	work.
	The syntax for referring to a \texttt{int} type is: \texttt{int}.
	
	The syntax for an integer value follows the syntax for C-style integer
	literals; base 8, 10, and 16 and their prefixes are supported.
	
	\texttt{int}s can be added ('\texttt{+}'),
	subtracted ('\texttt{-}'), multiplied ('\texttt{*}'), and
	divided ('\texttt{/}') into each
	other. Exponentiation is also supported
	with the '\texttt{**}' operator. Bitshifting (left and right) can be done
	using the '\texttt{<<}' and '\texttt{>>}' operators respectively.
	Bitwise and-ing, or-ing, and xor-ing can be done with the '\texttt{\&}',
	'\texttt{|}' and '\texttt{\^}' operators respectively.
	
	Equal-to, not-equal-to, greater-than, greater-than-equal-to,
	less-than, less-than-equal-to comparisons can be done on integer values with
	the '\texttt{==}', '\texttt{!=}',
	'\texttt{>}', '\texttt{>=}', '\texttt{<}', '\texttt{<=}' respectively.
	
	\texttt{int}s can be converted into floats using the \texttt{float!}
	special form.
}
