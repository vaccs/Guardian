

\section{Grammar Rules}
{
	Grammar rules function as a kind of building-block to connect to one
	other in order to describe the syntactic structure of a file to be parsed.
	Due to Guardian's amount of grammar-rule options, an entire file
	may be able to be described by only one grammar rule and still be quite
	complex.
	The entire content of the file must be covered (or perhaps "explained") by
	the structure described by the grammar rules.
	
	Different parser generators sometimes have different interpretations of
	what exactly is a grammar rule. In Guardian's case, the user can think
	of a grammar rule as a regular expression of tokens. A token is a series of
	consecutive characters read from the file.
	In a sense, a grammar rule is a regular expression of regular expressions.
	
	Grammar rules can build off of each other by including one another in their
	patterns.
	Grammar rules can be associated with a name, and that name can be used in
	other grammar rules. Grammar rules can refer to their own name.
	
	The initial grammar rule that begins the parse is referred to as the
	"start" rule, and is given in the \texttt{\%parse} directive.
	
	Remember that if the \texttt{\%skip} directive was used, there can be
	any number of silently-ignored sequences of characters in the file matching
	the \texttt{\%skip} pattern between
	(but not inside of) the strings matched as tokens.
	
	Once the parser has completed parsing the file, the parsed data is available
	to Guardian's programming language. This data takes the form of
	interconnected grammar-rule \textit{objects}, each instance mapping to a
	specific moment the parser completed parsing a grammar rule.
	Certain components of a grammar rule pattern can be named, these names
	are used as the field names of the created object to refer to the data of
	that component. In the case of a named token, the data of the field typically
	refers to the consecutive characters that make up that token in the form
	of a string value.
	In the case of a named reference to another grammar rule, the data of the
	field is the instance of the referred grammar rule's object.
	
	If the pattern of a grammar-rule can have more than one way of matching
	the same file content, where one possible way would result in a named
	component getting parsed (and therefore the corresponding object-value
	instance gaining a new field) and another possible way would result in the
	named component \textit{not} getting parsed, Guardian would prefer the
	way that would yield the most amount of named fields being parsed.
	In this sense, Guardian's file parser is "greedy".
	
	Depending on the grammar-rule pattern, the same component may be read
	multiple times in the same grammar-pattern parsing instance. If the
	component is named, the user can specify that either all of the values
	of that component for all the times it was read or if
	only the value for the \textit{first} time the component was read
	should be kept.
	
	Grammar rules can be named by inserting a name and a colon before the pattern:
	\begin{lstlisting}[numbers = none, texcl = true, language = MAIA]
<name>: <grammar-rule-pattern>;
	\end{lstlisting}
	
	The syntax for naming a component of a grammar-rule pattern is
	\texttt{\#<name>}.
	It should be inserted directly after the component in the pattern.
	If, due to the structure of the pattern, the component is read more than
	once, the default behavior is to keep only the first value. Insert open
	and close square brackets after the name (\texttt{\#<name>[]}) to indicate
	that all of the values should be kept. This field would then appear to the
	programming language a list where the value of each time the component was
	read is an element.
	
	A more detailed explanation of the syntax of grammar rule patterns follow.
	Parenthesis can be used to raise the precedence of a regular expression.
	
	\subimport{./}{1.character-literals.tex}
	
	\subimport{./}{2.integer-literals.tex}
	
	\subimport{./}{3.string-literals.tex}
	
	\subimport{./}{4.character-sets.tex}
	
	\subimport{./}{5.regular-expressions.tex}
	
	\subimport{./}{6.suffix-operators.tex}
	
	\subimport{./}{7.or-operator.tex}
	
	\subimport{./}{8.examples.tex}
}
















