
\chapter{Input File Specification}
{
	A high-level description of the \textit{functionality} of Guardian, separate
	from its intended application, is that Guardian generates programs that
	parse data out of files, manipulate that parsed data, and check
	properties of the data to see whether certain assertions are true.
	Accordingly, Guardian's input files
	consist of parse statements, skip directives, grammar rules and regular
	expressions (for
	parsing data out of files, and for describing the syntactic structure of
	the file to be parsed); expressions and variable declarations
	(for manipulating data); and assertion
	statements (for---you guessed it---making assertions about the data).
	
	Guardian input files can include other files. Guardian treats the
	include directive as if the content of the referred file was pasted
	directly in the file doing the including. A file that is included is
	only read
	once\footnote{Determining whether a given path refers to a file that has
	already been read is done by using the \texttt{stat} system
	call's \texttt{st\_dev} and \texttt{st\_ino} fields.}.
	
	To aid in debugging the checkers (and Guardian), print
	statements can be used to print the value of variables or
	results of expressions. To suppress print statements when running
	checkers use
	the \texttt{-q} command-line flag.
	
	The sequence in which these statements
	are listed in the input file determines the
	order they are executed in in the generated checker program.
	
	\subimport{./1.directives/}{index.tex}
	
	\subimport{./2.expressions/}{index.tex}
	
	\subimport{./3.grammar-rules/}{index.tex}
	
	\subimport{./4.regular-expressions/}{index.tex}
	
	\subimport{./5.character-sets/}{index.tex}
}










