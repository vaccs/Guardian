
\chapter{Input File Specification}
{
	A high-level description of the \textit{functionality} of Guardian, separate
	from its intended application, is that Guardian generates programs that
	parse data out of files, manipulate the parsed data, and checks/asserts whether
	certain properties about that data are true. As such, Guardian input files
	consist of: parse statements, skip directives and grammar rules, (for
	parsing data out of files, and for describing the syntactic structure of
	the file to be parsed); expressions and variable declarations; (for data
	manipulation); and assertion statements. (for --you guessed it-- making
	assertions about the data).
	
	Guardian input files can also include other files. Guardian treats this as
	if the content of the referred file was pasted directly in place of where
	the include directive has occurred. An included file is only read
	once\footnote{Determining wether a given path refers to a file that has
	already been read is done by using st\_dev and st\_ino}.
	
	For sake of making debugging the checkers (and Guardian) easier, print
	statements can be used to print the value of variables or
	results of expressions. Print statements are by default disabled and can be
	enabled by invoking the generated checker with the '-v' command-line flag.
	
	The sequence that these statements and directives are listed in an input
	file will match the
	order they are executed in the generated checker program.
	
	\subimport{./1.directives/}{index.tex}
	
	\subimport{./2.grammar-rules/}{index.tex}
	
	\subimport{./3.regular-expressions/}{index.tex}
	
	\subimport{./4.character-sets/}{index.tex}
	
	\subimport{./5.expressions/}{index.tex}
}










