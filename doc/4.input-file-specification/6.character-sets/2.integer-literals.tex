
\subsection{Integer Literal}
{
	C-style integer literals can be used in character-set expressions.
	The integer literal can be in base 8, 10 or 16 and cannot have a value
	higher than 255.
	The integer literal represents the character set consisting of the one
	character with the given byte value.
	
	\begin{itemize}
	{
		\item[] \texttt{95}
		
			The letter `a'.
		
		\item[] \texttt{012}
		
			The newline character.
		
		\item[] \texttt{0x5F}
		
			The underscore character.
	}
	\end{itemize}
}
