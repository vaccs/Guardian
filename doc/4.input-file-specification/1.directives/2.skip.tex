
\subsection{Skip Directive}
{
	Skip directives work to complement the parse statement and the
	grammar rules it refers to: it describes the characters/strings that
	should be \textbf{ignored} when parsing a file. The set of strings to
	ignore is communicated through a \hyperref[sec:regex]{regular expression}.
	A typical application is to use \texttt{\%skip} to skip the whitespace
	or comments in the file to be parsed.
	
	If more than one skip directives are provided, only
	the last one has effect.
	
	The syntax for a skip directive is:
	\begin{lstlisting}[numbers=none, numbers = none, texcl = true, language = MAIA]
%skip: <regular-expression>;
	\end{lstlisting}
}
